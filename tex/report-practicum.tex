\documentclass[11pt]{article}
\usepackage[utf8]{inputenc}
\usepackage[russian]{babel}
\usepackage{amssymb}
\usepackage{amsmath}
\newcommand{\w}{\widetilde}
\voffset = -2cm 
\footskip=200pt
\begin{document}
\begin{center}

\textsc{Численное моделирование нестационарного 
    одномерного течения газа с использованием схемы 
    для логарифма плотности с центральными разностями}

\end{center}

\enlargethispage{9\baselineskip}

\section{Постановка дифференциальной задачи}
Одномерное движение вязкого баротропного газа описывается системой дифференциальных уравнений:
$$
\begin{cases} 
\displaystyle{\frac{\partial \rho}{\partial t} +
 \frac{\partial \rho u}{\partial x} = 0}, \\

\displaystyle{\rho \frac{\partial u}{\partial t} +
 \rho u \frac{\partial u}{\partial x}+ 
 \frac{\partial p}{\partial x} = 
\mu \frac{\partial^2 u}{\partial x^2}
+ \rho f}
\end{cases}
$$

\begin{center}
$
p = p (\rho)
$ - известная функция давления от плотности
\end{center}
$$\mu \in [0,001; 0,1]$$

Неизвестные функции плотности и скорости:

\begin{center}
$\rho,\ u: [0, T] \times [0, X] \rightarrow \mathbb{R}$, где $T,\ X > 0$
\end{center}

$$\rho > 0$$

Краевые условия:
$$(\rho, u)|_{t=0} = (\rho_0, u_0), x \in [0, X]$$

Для гарантирования положительности $\rho$ вместо функции $\rho$ имеет смысл искать функцию $g = \ln(\rho)$ 

Тогда система дифференциальных уравнений примет вид

$$
\begin{cases}
\displaystyle{\frac{\partial g}{\partial t} + \frac{1}{2} \left(u\frac{\partial g}{\partial x} +     \frac{\partial ug}{\partial x} +(2-g)\frac{\partial u}{\partial x}\right) = 0},\\
\displaystyle{\frac{\partial u}
{\partial t} + \frac{1}{3} \left(u 
\frac{\partial u}{\partial x} +
 \frac{\partial u^2}{\partial x}\right) +
  \tilde{p}'(g)
 \frac{\partial g}{\partial x} = \mu
  e^{-g}\frac{\partial^2 u}{\partial x^2} +
   f}
\end{cases}
$$

\end{document}